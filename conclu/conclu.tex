This study provides a comprehensive analysis of the entanglement properties in both 2-qubit and 3-qubit systems modeled by the Heisenberg XY Hamiltonian with Dzyaloshinskii-Moriya interaction (DMI). The key findings from our analysis can be summarized as follows:

\begin{enumerate}
    \item \textbf{Impact of Dzyaloshinskii-Moriya Interaction (DMI):}
    The presence of the DMI, represented by the parameter \(D\), consistently enhances the entanglement between qubits. This is observed through increased concurrence values in both the 2-qubit and 3-qubit systems when \(D = 1\) compared to when \(D = 0\). The DMI facilitates stronger quantum correlations, making it a valuable tool for increasing entanglement in quantum systems.

    \item \textbf{Effect of Anisotropy Parameter \(\delta\):}
    The anisotropy parameter \(\delta\) introduces a counteracting influence on the entanglement. While the system shows high concurrence values when \(\delta = 0\), an increase in \(\delta\) generally leads to a reduction in these values, especially in the presence of DMI. This suggests that \(\delta\) affects the symmetry and interaction strengths within the lattice, reducing the overall entanglement when increased.

    \item \textbf{Concurrence in Different Configurations:}
    The study demonstrates that, in the absence of DMI (\(D = 0\)), the system still exhibits significant entanglement, particularly when \(\gamma = 1\) and \(\delta = 0\). In such cases, concurrence values can reach up to approximately 0.81 in 2-qubit systems and 0.8 in 3-qubit systems. This indicates that the Heisenberg XY model, even without DMI, supports robust quantum entanglement.

    \item \textbf{Tuning Entanglement through Parameters:}
    The results highlight the importance of the interplay between \(D\), \(\delta\), and \(\gamma\) in tuning the entanglement properties of quantum systems. By carefully adjusting these parameters, one can control the degree of quantum entanglement, which is essential for optimizing quantum information processing tasks and designing advanced quantum materials.
\end{enumerate}

In conclusion, the Dzyaloshinskii-Moriya interaction and the anisotropy parameter provide powerful mechanisms for manipulating entanglement in quantum systems. Understanding how these parameters influence entanglement allows for the precise control of quantum correlations, which has significant implications for the development of quantum technologies. This study lays the groundwork for future research into optimizing entanglement in more complex quantum systems and exploring the practical applications of these findings in quantum computing and materials science.