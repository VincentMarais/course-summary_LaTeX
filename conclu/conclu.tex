This study provides a comprehensive 
analysis of the entanglement properties 
in both 2-qubit, 3-qubit and 6-qubit systems modeled 
by the Heisenberg XY Hamiltonian with 
Dzyaloshinskii-Moriya interaction (DMI). 
The key findings from our analysis 
can be summarized as follows:

\begin{enumerate}
    \item \textbf{Impact of Dzyaloshinskii-Moriya Interaction (DMI):}
    The presence of the DMI, represented by the parameter \(D\), consistently enhances the entanglement between qubits. This is observed through increased concurrence values in both the 2-qubit, 3-qubit and 
    6-qubit systems when \(D = 1\) compared to when \(D = 0\). The DMI facilitates stronger quantum correlations, making it a valuable tool for increasing entanglement in quantum systems.

    \item \textbf{Effect of Anisotropy Parameter \(\delta\):}
    The anisotropy parameter \(\delta\) introduces a counteracting influence on the entanglement. While the system shows high concurrence values when \(\delta = 0\), an increase in \(\delta\) generally leads to a reduction in these values, especially in the presence of DMI. This suggests that \(\delta\) affects the symmetry and interaction strengths within the lattice, reducing the overall entanglement when increased.

\end{enumerate}

In conclusion, the Dzyaloshinskii-Moriya interaction 
and the anisotropy parameter provide powerful 
mechanisms for manipulating entanglement in 
quantum systems. Understanding how these parameters 
influence entanglement allows for the precise control of the 
quantum system, which has significant implications for the development of quantum technologies. 
This study lays the groundwork 
for future research into optimizing 
entanglement in more complex quantum systems and exploring the practical applications of these findings in 
quantum computing and materials science.