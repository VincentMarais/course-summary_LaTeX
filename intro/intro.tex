Quantum entanglement, a fundamental feature of quantum mechanics, 
was first recognized in the early 20th century by Einstein, 
Podolsky, Rosen, and Schrödinger \cite{horodecki_quantum_2009}. This phenomenon describes 
a situation where the quantum states of two or more particles 
become intertwined, such that the state of one particle cannot 
be described independently of the state of the other, 
no matter the distance separating them. 



Entanglement become a practical resource 
in quantum information science, 
underpinning technologies such as quantum cryptography \cite{pirandola_advances_2020}, 
quantum teleportation \cite{bennett_teleporting_1993}, and quantum computing \cite{whitfield_quantum_2022}. 
These applications exploit entanglement 
to perform tasks that are impossible 
with classical systems.

Despite its utility, entanglement is a fragile and complex phenomenon, 
challenging to detect and manipulate. 
The study of entanglement involves understanding 
its properties, methods for its detection, and 
strategies for its quantification and manipulation. 
These efforts are crucial for advancing our ability 
to harness entanglement for practical applications, 
ensuring that it can be effectively used as a resource
in quantum communication and computation.



The work presented a quantum simulation of a specific
model of 1D lattice of \textbf{qubits} (quantum bit) a quantum system 
capable of existing in two states, such as 
the spin-up \(\ket{\uparrow}\) and spin-down 
\(\ket{\downarrow}\) states of an electron call Heisenberg XY model \cite{van_der_sijs_heisenberg_1993}
and with a Dzyaloshinskii-Moriya interaction (DMI) \cite{moriya_anisotropic_1960,dzyaloshinsky_thermodynamic_1958}. Finally 
a protocol to accelerating the entanglement in this lattice. 


