Quantum entanglement, a fundamental feature of quantum mechanics, 
was first recognized in the early 20th century by Einstein, 
Podolsky, Rosen, and Schrödinger \cite{horodecki_quantum_2009}. This phenomenon describes 
a situation where the quantum states of two or more particles 
become intertwined, such that the state of one particle cannot 
be described independently of the state of the other, 
no matter the distance separating them. 



Entanglement become a practical resource 
in quantum information science, 
underpinning technologies such as quantum cryptography \cite{pirandola_advances_2020}, 
quantum teleportation \cite{bennett_teleporting_1993}, and quantum computing \cite{whitfield_quantum_2022}. 
These applications exploit entanglement 
to perform tasks that are impossible 
with classical systems.

Despite its utility, entanglement is a fragile and complex phenomenon, 
challenging to detect and manipulate. 
The study of entanglement involves understanding 
its properties, methods for its detection, and 
strategies for its quantification and manipulation. 
These efforts are crucial for advancing our ability 
to harness entanglement for practical applications, 
ensuring that it can be effectively used as a resource
in quantum communication and computation.

The work presented here go will study the simplification of the Kitaev model call $XY-\Gamma$ model 
because this model have a good futur to build a quantum computer (cf paper XY-Gamma for the ref about Kitaev) 
I this model we will see how the entanglement propagate and quantify the entanglement.

\subsection{Model XY-DMI}
Introduction on the Heinsenberg model + 
Introduction on the DMI + spin chain +qubits



The XY-$\Gamma$ model \cite{kheiri_information_2024} or XY model is a devivative of the 

with a Dzyaloshinskii–Moriya interaction (DMI) \cite{moriya_anisotropic_1960,dzyaloshinsky_thermodynamic_1958} is one-dimentionel spin chain. 


\begin{equation}
	\mathcal{\hat{H}} = \mathcal{\hat{H}}_{XY} + \mathcal{\hat{H}}_{\Gamma},
\end{equation}
where
\begin{equation}
	\mathcal{\hat{H}}_{XY} = J \sum_{n=1}^L \left[ \left( \frac{1 + \delta}{2} \right) \sigma^x_n \sigma^x_{n+1} + \left( \frac{1 - \delta}{2} \right) \sigma^y_n \sigma^y_{n+1} \right] + h \sum_{n=1}^L \sigma^z_n,
\end{equation}
and
\begin{equation}
	\mathcal{\hat{H}}_{\Gamma} = \Gamma \sum_{n=1}^L \left( \sigma^x_n \sigma^y_{n+1} + \gamma \sigma^y_n \sigma^x_{n+1} \right).
\end{equation}

Here, $\sigma^x, \sigma^y, \sigma^z$ are the Pauli matrices, $J$ is the exchange constants, 
$\delta$ is the anisotropy parameter, 
$h$ is the strength of the transverse field, and 
$D$ characterizes the amplitude of the off-diagonal exchange interactions with 
$\gamma$ being the relative coefficient of these couplings.

eff

\begin{itemize}
    \item \textbf{XY term} : $\hamil_{XY}$ describes the nearest-neighbor interactions along the 
	$x$ and $y$ directions with anisotropy $\delta$ and a transverse magnetic field $h$.
	\item \textbf{$\Gamma$ term or Dzyaloshinskii-Moriya interaction} :
	 $\hamil_{\Gamma}$ introduces an additional interaction 
	that mixes the $x$ and $y$ components of neighboring spins, breaking mirror symmetry (Because the particle at 
	position $n$ does not have the same orientation of spin as the particle at position $n+1$, 
	as represented by the term $\sigma^x_n \sigma^y_{n+1}$, where the particle at position $n$ has a spin oriented in the $x$ 
	direction and the particle at position $n+1$ has a spin oriented in the $y$ direction.
	).
\end{itemize}



\subsection{Entanglement metric} 

cf the video \url{}