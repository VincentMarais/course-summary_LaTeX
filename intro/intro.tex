






\subsection{Model $\mathbf{XY-\Gamma}$}

The XY-$\Gamma$ model is an extension of the one-dimensional XY spin-1/2 model, 
incorporating an additional symmetric off-diagonal interaction term denoted by $\Gamma$. 
The Hamiltonian of this model is given by:

\begin{equation}
	\mathcal{\hat{H}} = \mathcal{\hat{H}}_{XY} + \mathcal{\hat{H}}_{\Gamma},
\end{equation}
where
\begin{equation}
	\mathcal{\hat{H}}_{XY} = J \sum_{n=1}^L \left[ \left( \frac{1 + \delta}{2} \right) \sigma^x_n \sigma^x_{n+1} + \left( \frac{1 - \delta}{2} \right) \sigma^y_n \sigma^y_{n+1} \right] + h \sum_{n=1}^L \sigma^z_n,
\end{equation}
and
\begin{equation}
	\mathcal{\hat{H}}_{\Gamma} = \Gamma \sum_{n=1}^L \left( \sigma^x_n \sigma^y_{n+1} + \gamma \sigma^y_n \sigma^x_{n+1} \right).
\end{equation}

Here, $\sigma^x, \sigma^y, \sigma^z$ are the Pauli matrices, $J$ is the exchange constants, 
$\delta$ is the anisotropy parameter, 
$h$ is the strength of the transverse field, and 
$\Gamma$ characterizes the amplitude of the off-diagonal exchange interactions with 
$\gamma$ being the relative coefficient of these couplings.

eff
\subsection{Physical Interpretation}

\subsubsection{Interactions and Terms}
\begin{itemize}
    \item \textbf{XY term} : $\hamil_{XY}$ describes the nearest-neighbor interactions along the 
	$x$ and $y$ directions with anisotropy $\delta$ and a transverse magnetic field $h$.
	\item \textbf{$\Gamma$ term or Dzyaloshinskii-Moriya interaction} :
	 $\hamil_{\Gamma}$ introduces an additional interaction 
	that mixes the $x$ and $y$ components of neighboring spins, breaking mirror symmetry (Because the particle at 
	position $n$ does not have the same orientation of spin as the particle at position $n+1$, 
	as represented by the term $\sigma^x_n \sigma^y_{n+1}$, where the particle at position $n$ has a spin oriented in the $x$ 
	direction and the particle at position $n+1$ has a spin oriented in the $y$ direction.
	).
\end{itemize}

\subsubsection{Magnetocrystalline anisotropy}
Explaintion of the anisotropy term $\delta$  :

Magnetocrystalline anisotropy arises from the crystalline structure of a material, influencing 
the preferred direction of magnetization. This direction varies based on the material's lattice structure; 
for example, iron with a cubic lattice favors the ±xyz directions, while nickel prefers diagonal directions. 
The phenomenon is attributed to differing interaction strengths between neighboring lattice sites along various crystal planes. 
In polycrystalline films, composed of randomly oriented small clusters, the average magnetocrystalline anisotropy is negligible. 
Similarly, in amorphous films, where inter-atomic distances are random, this anisotropy can also be ignored.

Paper : Magnetic Anisotropy
Assistant: Yifan Zhou, +358-451345822

\subsubsection{Phases of the Model}
\begin{itemize}
    \item \textbf{Ferromagnetic (FM) Phase}: Characterized by aligned spins, typically occurring at low transverse field strength $h$.
    \item \textbf{Paramagnetic (PM) Phase}: Spins are disordered due to a strong transverse field.
    \item \textbf{Spiral Phase}: Exhibits a quasi-long-range order, with spins forming a spiral pattern. This phase emerges due to the interplay between the XY interactions and the $\Gamma$ term.
\end{itemize}

\subsection{Information Propagation}
\begin{itemize}
    \item The $\Gamma$ interaction affects the way information spreads through the system. It creates an asymmetric "light-cone" structure with different propagation speeds (butterfly velocities) for information in different directions.
    \item In the spiral phase, information propagates faster compared to the FM and PM phases, where the propagation is slower.
\end{itemize}

\subsection{Applications} 

cf Paper Simulation of XY model in a quantum computer
Author: Marc Farreras Bartra




Paper ()