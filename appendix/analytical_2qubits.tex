The objective of this section is to solve the Schrödinger equation for a chain of two qubits:

\[ 	\left\{\begin{matrix}
		i \hbar \partial_t \ket{\psi(t)} = \hamil\ket{\psi(t)}  \\
		 \\
		 \ket*{\psi (t=0)} = \ket*{\up \down}
	   \end{matrix}\right.\]



The Hamiltonian of this system is given by:

\begin{equation}\label{eq: Hamiltionian model XY-Gamma 2 qubits}
	\hamil =  \hamil_{XY} + \hamil_{D}
\end{equation}

\begin{equation}\label{eq: Hamiltionian model XY 2 qubits}
	\hamil_{XY} = \hbar J\left [ \left ( \frac{1 + \delta}{2} \right ) \sigma_x^{(1)} \otimes \sigma_x^{(2)} 
	+ \left ( \frac{1 - \delta}{2} \right ) \sigma_y^{(1)} \otimes \sigma_y^{(2)} \right ] 
	+ \hbar g \sigma_z^{(1)} \otimes \id{2}{2}
\end{equation}

and 

\begin{equation}\label{eq: Hamiltionian model Gamma 2 qubits}
	\hamil_{D} = D \hbar \left ( \sigma_x^{(1)} \otimes \sigma_y^{(2)} + \gamma \sigma_y^{(1)} \otimes \sigma_x^{(2)} \right ) 
\end{equation}
Now let define the Hilbert space of the quantum system of 
two qubits, first of all the Hilbert 
basis of the first qubit ($\mathcal{H}_1$: 
Hilbert space of the first qubit) is given by:

\[ \ket{\up}^{(1)} =  \begin{pmatrix} 1 \\0 \end{pmatrix} \ \text{and} \ \ket{\down}^{(1)} =  \begin{pmatrix} 0 \\1 \end{pmatrix} \]

then the Hilbert basis of the second qubit ($\mathcal{H}_2$: Hilbert space of the second qubit) is given by:

\[ \ket{\up}^{(2)} =  \begin{pmatrix} 1 \\0 \end{pmatrix} \ \text{and} \ \ket{\down}^{(2)} =  \begin{pmatrix} 0 \\1 \end{pmatrix} \]

Therefore the basis for the total Hilbert space ($\mathcal{H}_{tot} = \mathcal{H}_{1} \otimes \mathcal{H}_{2}$) is:

\[ \mathcal{B}_{\mathcal{H}_{tot}} = \{ \ket{\up}^{(1)} \otimes \ket{\up}^{(2)}, \ket{\up}^{(1)} \otimes \ket{\down}^{(2)}, \ket{\down}^{(1)} \otimes \ket{\up}^{(2)},  \ket{\down}^{(1)} \otimes \ket{\down}^{(2)} \} \]

To simplify the notation in this report, we will use a classical notation:

\begin{equation}\label{eq: basis H tot}
	\mathcal{B}_{\mathcal{H}_{tot}} = \{ \ket{\up \up}, \ket{\up \down}, \ket{\down \up}, \ket{\down \down} \} 
\end{equation}

The Hilbert space, 
a fundamental construct 
in quantum mechanics and quantum 
computing, represents the complete set of possible states 
for a quantum system. However, the vastness of this space 
often poses significant computational challenges. 
The reduction of Hilbert space, therefore, becomes 
a crucial technique for making quantum computations 
tractable. This process involves identifying and 
isolating a subspace of the Hilbert space that 
captures the essential dynamics and properties 
of the system under consideration.



\subsubsection{Hilbert space reduction}
To reduce of the Hilbert space. Let see the action of the 
$\hamil$ on the initial state

\begin{equation}\label{eq: action hamil on initial state}
    \hamil \ket*{\up \down} = \hbar \left[J + i D(\gamma - 1)\right] \ket*{\down \up} -
    + \hbar g  \left(\ket*{\up \down} - \ket*{\up \down} \right) 
\end{equation}



With \myeqref{eq: action hamil on initial state} the minimal 
basis of the system is :
\begin{equation}
    \basis_{red} = \{ \ket*{\up \down} , 
    \ket*{\down \up}  \}
\end{equation}

So let expressed the Hamiltionian in $\basis_{red}$, for that 
let see the action of $ \hamil $ on the $\ket*{\down \up}$

\begin{equation}
    \hamil \ket*{\down \up} = \hbar \left[J - i D(\gamma - 1)\right] \ket*{\up \down} 
    - \hbar  g\left(\ket*{\down \up} - \ket*{\down \up} \right) 
\end{equation}


So with the the effectif Hamiltionian in $\basis_{red}$ is :

\begin{equation}\label{eq: Heff in basis reduce 2 qubits}
    \boxed{\hamil_{\text{eff}} =  \hbar \begin{bmatrix}
        0 & \kappa  \\
        \kappa^* &  0\\
        \end{bmatrix} }
\end{equation}

where $\kappa = J + i D(\gamma - 1)$

Let us define $a = \frac{1+\delta}{2}$ and $b=\frac{1-\delta}{2}$, and 
let us transform the $\hamil_{XY}$ using \refprop{prop:Identity Pauli Matrix}:
\begin{equation*}
	\hamil_{XY} = \hbar J \left[ a\left( \splus{1} 
	+ \sminus{1}   \right) \left( \splus{2} 
	+ \sminus{2}  \right) - b \left( \splus{1} 
	- \sminus{1}   \right) \left( \splus{2} 
	- \sminus{2}  \right) \right] + 
	\hbar g \sigma_z^{(1)}
\end{equation*}



Now, let us use the bilinearity and associativity of the Kronecker product:

\begin{align*}
    \hamil_{XY} &= \hbar J \left[ a \left( 
        \splus{1} \splus{2} 
        + \splus{1} \sminus{2} 
        + \sminus{1} \splus{2}
        + \sminus{1} \sminus{2} \right) \right. \\ 
    &\quad \left. - b \left( 
        \splus{1} \splus{2} 
        - \splus{1} \sminus{2} 
        - \sminus{1} \splus{2}
        + \sminus{1} \sminus{2} \right) \right] 
    + \hbar g \left[ \sz{1} + \sz{2}\right]
\end{align*}

Since $a + b = 1$ and $a-b = \delta $, we have:

\begin{align*}
		\hamil_{XY} = \hbar J \left[ \delta \left( 
		\splus{1} \splus{2} 
		+ \sminus{1} \sminus{2} \right) 
		+ \left(
		\splus{1} \sminus{2} 
		+ \sminus{1} \splus{2}
		\right)   \right] 
		+ \hbar g \left[ \sz{1} + \sz{2}\right]
\end{align*}

Let us now proceed similarly for $\hamil_{D}$:

\begin{align*}
	\hamil_{D} &= \hbar D \left( 
	\sx{1} \sy{2} + \gamma \sy{1} \sx{2}
	\right) \\
	&= - i\hbar D 
	\left[ \left(\splus{1} + \sminus{1}  \right) 
	\left(\splus{2} - \sminus{2}  \right) + \gamma
	\left(\splus{1} - \sminus{1}  \right) 
	\left(\splus{2} + \sminus{2}  \right) \right]
	\ \ \ \  
	\text{by \refprop{prop:Identity Pauli Matrix}} \\
	&=-i\hbar D 
	\left[ \left(
	\splus{1} \splus{2} -
	\splus{1} \sminus{2} +
	\sminus{1}  \splus{2} -
	\sminus{1} \sminus{2} \right) 
	+ \gamma
	\left(\splus{1} \splus{2} +
	\splus{1} \sminus{2} -
	\sminus{1}  \splus{2} -
	\sminus{1} \sminus{2} \right)
	\right] \\
	&=
	-i\hbar D 
	\left[ (\gamma + 1) 
	\splus{1} \splus{2} +
	(\gamma - 1) \splus{1} \sminus{2} -
	(\gamma-1) \sminus{1}  \splus{2} 
	-(\gamma + 1)\sminus{1} \sminus{2} 
	\right] \\
	&=
	i\hbar D 
	\left[ \left(\gamma + 1 \right) \left(\sminus{1} \sminus{2}
	- \splus{1} \splus{2}  \right)  + 
	\left(\gamma - 1 \right) \left(
		\sminus{1} \splus{2} - 
		\splus{1} \sminus{2}\right) 
	\right]
\end{align*}

Given the initial state $\ket*{\uparrow \downarrow}$, as discussed in the section on reduction of Hilbert space, we can reduce the basis of the Hilbert space to $\basis_{red}$. The operators are as follows:



\begin{equation}
	\left\{\begin{matrix}
		i\left(\sminus{1} \splus{2} - 
		\splus{1} \sminus{2} |_{\basis_{red}}\right) = \sigma_y 	\\
		\\
		\splus{1} \sminus{2} 
		+ \sminus{1} \splus{2} |_{\basis_{red}} = \sigma_x \\
		\\
		\sz{1}|_{\basis_{red}} = \sigma_z \\
		\\

		\sz{2}|_{\basis_{red}} = -\sigma_z \\
		\\
		\splus{1} \splus{2} 
		+ \sminus{1} \sminus{2} |_{\basis_{red}}
		= \sminus{1} \sminus{2}
		- \splus{1} \splus{2} |_{\basis_{red}} 
		= 0
	   \end{matrix}\right.
\end{equation}

\begin{equation*}
	\hamil_{\text{eff}} = \hbar \left( 
		J \sigma_x + D(\gamma - 1)\sigma_y + g \sigma_z
	 \right)
\end{equation*}



Using \refth{thm:Matrix exponential and Pauli matrices} and \myeqref{eq: Heff in basis reduce 2 qubits}, the evolution operator in the basis $\basis_{red}$ is given by:

\begin{equation}
	\hat{U}(t,t_0) = e^{-i\hamil_{\text{eff}} t /\hbar} = \cos(\mu t) \hat{\mathbb{I}} 
	- i \sin(\mu t) \left[ \frac{J \sigma_x + D(\gamma - 1)\sigma_y}{\mu} \right]
\end{equation}


where $\mu = \sqrt{J^2 + D^2(\gamma-1)^2 }$.

Thus, using the definition of the time evolution operator and the properties in \refprop{prop:Action of Pauli matrices on spin states}, we can compute $\ket*{\psi (t)}$:

\begin{equation}\label{eq: psi t}
	\boxed{	\ket{\psi (t)} = \cos(\mu t)  \ket*{\uparrow \downarrow} -i\sin(\mu t) \left[ \frac{J +i D (\gamma -1)}{\mu}  \right] \ket*{\downarrow \uparrow}
	}
\end{equation}

\myth{Concurrence of two qubit}{
	
	Consider a two-qubit system with the basis states $\mathcal{B} = \{ \ket{\downarrow \downarrow} , \ket{\downarrow \uparrow}, 
	\ket{\uparrow \downarrow}, \ket{\uparrow \uparrow} \}$ for its Hilbert space. 
	
	For a quantum state $\ket{\psi}$ in $\mathcal{B}$ expressed as 
	\begin{equation}\label{eq: psi qubit}
		\ket{\psi} = C_0 \ket{\downarrow \downarrow} + C_1\ket{\downarrow \uparrow} + C_2\ket{\uparrow \downarrow} +  C_3\ket{\uparrow \uparrow}  
	\end{equation}
	where $\forall i \in [[0, 3]], \ C_i \in \mathbb{C}$
	the concurrence $C(\ket{\psi})$ is given by
	
	
	\begin{equation}
		C\left( \ket{\psi} \right)  = 2\left| C_1C_2 - C_0C_3 \right| 
	\end{equation}
	
}{Concurrence of two qubit}	




\begin{proof}
	We start by recalling the definition of concurrence $C(\ket{\psi})$ for a two-qubit quantum state $\ket{\psi}$ 
	\cite{wootters_entanglement_1998}:

\begin{equation}
	C\left( \ket{\psi} \right)  = \left|  \bra{\psi} \sigma_y \otimes \sigma_y \ket{\psi^*} \right| 
\end{equation} 



The action of the tensor product $\sigma_y \otimes \sigma_y$ on $\ket{\psi^*}$, considering the transformation properties of $\sigma_y$ on the basis states, is computed as:



\begin{equation}
	\sigma_y \otimes \sigma_y \ket{\psi^*}= C_0^* \ket{\uparrow \uparrow}  + C_1^* \ket{\uparrow \downarrow} + C_2^* \ket{\downarrow \uparrow}  +  C_3^*\ket{\downarrow \downarrow}  
\end{equation}

Next, we evaluate the inner product $\bra{\psi} \sigma_y \otimes \sigma_y \ket{\psi^*}$. Recall that the basis states are orthogonal and normalized. Therefore, we only consider terms where the bra and ket vectors match, which gives:





\begin{equation}
	\bra{\psi} \sigma_y \otimes \sigma_y \ket{\psi}^* = -2C_0C_3 + 2C_1C_2
\end{equation}

Thus, the concurrence $C(\ket{\psi})$ becomes:


\begin{equation*}
	C\left( \ket{\psi} \right)  = 2 \left|  C_1C_2 -C_0C_3 \right| 
\end{equation*}
\end{proof}

So 
\begin{align*}
	C(\ket*{\psi(t)}) &= 2\abs{ -\cos(\mu t)	
	i\sin(\mu t) \left[ \frac{J 
	+i D (\gamma -1)}{\mu}  \right]  } \\
	&= \abs*{\sin(2\mu t)}
\end{align*}

\begin{equation*}
	\boxed{C(\ket*{\psi(t)}) = \abs*{\sin(2 \mu t)}}
\end{equation*}
