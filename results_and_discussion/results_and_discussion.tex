Results And Discussion


\subsection{Entanglement in a spin chain with 2 qubits}
In this section, we analyze the dynamics of a 
system consisting of two qubits with the Hamiltonian given by:

\[
    \hat{H} =  \hbar J\left [ \left ( \frac{1 + \delta}{2} \right ) \sigma_1^x \otimes 
    \sigma_2^x  + \left ( \frac{1 - \delta}{2} \right ) \sigma_1^y \otimes \sigma_2^y \right ] + 
    D \hbar \left ( \sigma_1^x \otimes \sigma_2^y + \gamma \sigma_1^y \otimes \sigma_2^x  \right ) + g \left( 
    \sigma_1^z \otimes \mathbb{\hat{I}} + \mathbb{\hat{I}} \otimes \sigma_2^z  
\right).
\]

we focus on the concurrence $C(\rho(t))$, which quantifies the entanglement between 
the two qubits. The analytical solution for the concurrence, as derived in Appendix B, is given by:

\begin{equation*}
        C(\rho(t)) = \abs*{\sin(2 \mu t)},
\end{equation*}

where \(\mu = \sqrt{J^2 + D^2(\gamma-1)^2}\).

\subsubsection{Analysis of Concurrence Dynamics}

The concurrence formula reveals a simple yet profound oscillatory behavior of the entanglement between the qubits. The parameter \(\mu\) plays a crucial role in determining the frequency of these oscillations, which depends on the coupling constants \(J\) and \(D\) as well as the anisotropy parameter \(\gamma\).

\textbf{Impact of Coupling Constants \(J\) and \(D\):}

The coupling constant \(J\) directly contributes to \(\mu\), indicating that a stronger XX and YY interaction 
        leads to faster oscillations in concurrence. The term \(D^2(\gamma-1)^2\) suggests 
        that the effect of the \(D\) coupling on the concurrence 
        is modulated by the anisotropy \(\gamma\). For \(\gamma = 1\), 
        this contribution vanishes, and the oscillation frequency is 
        solely determined by \(J\). However, for \(\gamma \neq 1\), 
        the anisotropy introduces additional dynamics through \(D\).


 \textbf{Behavior for Different Regimes of \(\mu\):}

 When \(\mu\) is large (e.g., large \(J\) 
        or significant anisotropy), the concurrence oscillates 
        rapidly, meaning that the system frequently transitions 
        between entangled and separable states. For small \(\mu\), the oscillations are slower, 
        indicating more prolonged periods of either high or low entanglement.

\textbf{Maximal Concurrence:}
 The concurrence achieves its maximum value of 1 when \(\sin(2 \mu t) = \pm 1\), indicating a fully entangled state.
 Conversely, \(C(\ket*{\psi(t)}) = 0\) when \(\sin(2 \mu t) = 0\), representing separable states where the qubits are not entangled.

\subsection{Discussion}

The results provide significant insights into the entanglement 
dynamics in a two-qubit system with anisotropic and cross-coupling terms. 
The oscillatory nature of the concurrence reflects the intricate interplay 
between different types of interactions present in the Hamiltonian. 
This behavior is essential for applications in quantum information 
processing, where control over entanglement is crucial.

\begin{itemize}
    \item \textbf{Quantum Control:} By tuning the parameters 
    \(J\), \(D\), and \(\gamma\), one can manipulate 
    the entanglement dynamics, potentially allowing for 
    the design of specific quantum gates or the 
    implementation of quantum error correction protocols that rely on dynamic entanglement.

    \item \textbf{Effect of Anisotropy:} The dependence of \(\mu\) on 
    \(\gamma\) illustrates how anisotropy can either enhance or diminish 
    the contribution of the \(D\) coupling to the entanglement dynamics. 
    This result suggests that systems with tunable anisotropy could be 
    particularly versatile in quantum control schemes.

    \item \textbf{Robustness of Entanglement:} The periodic nature of the concurrence indicates that, under specific conditions, entanglement can be robust over time, recurring predictably as a function of time. This property might be exploited to maintain entanglement over long durations in quantum communication protocols.
\end{itemize}

In Conclusion
 concurrence 
 quantifies the entanglement between two qubits. 
 For example, when \(t/T_0 = 0.5 + k\), where \(k \in \mathbb{N}\), the concurrence is 1. At this time, the state \(\ket{\psi (t)}\) is:


 
 \begin{equation*}
 	\ket{\psi \left( t = \frac{\pi}{4} + \frac{k\pi}{2}\right) } = \frac{1}{\sqrt{2}} \ket{\uparrow \downarrow} - \frac{i(J + 2iD)}{\sqrt{2}\gamma}   \ket{\downarrow \uparrow}
 \end{equation*}
 
 
This state is evidently entangled according to the definition.

Moreover we can recognize a Bell state we this form: 

\begin{equation}
		\ket{\psi \left( t = \frac{\pi}{4} + \frac{k\pi}{2}\right) } = \frac{1}{\sqrt{2}} \left(  \ket{\uparrow \downarrow} -ie^{i\theta}\ket{\uparrow \downarrow}  \right)
\end{equation}

where $ \theta = \arctan\left( \frac{2D}{J}\right)$


 
And at $t/scale = k$ the concurrence equal to $0$, 

 \begin{equation*}
	\ket{\psi \left( t = \frac{k\pi}{2}\right) } = \frac{i(J + 2iD)}{\sqrt{2}\gamma}   \ket{\downarrow \uparrow}
\end{equation*}

In this case, the state is separable.
	

\includegraphics*{results_and_discussion/2_qubits/up_down_without_ana.pdf}

\subsubsection{Analytical approch}
\includegraphics*{results_and_discussion/2_qubits/up_down_with_ana.pdf}


\subsection{Entanglement in a spin chain with N qubits}

\subsubsection{Entanglement in a spin chain with 3 qubits}


\subsubsection{Entanglement in a spin chain with 8 qubits}


\subsection{Protocol of accelerating of the entanglement}